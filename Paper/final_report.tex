\documentclass[10pt,twocolumn,letterpaper]{article}

\usepackage{cvpr}
\usepackage{times}
\usepackage{epsfig}
\usepackage{graphicx}
\usepackage{amsmath}
\usepackage{amssymb}
\usepackage{float}




% Include other packages here, before hyperref.

% If you comment hyperref and then uncomment it, you should delete
% egpaper.aux before re-running latex.  (Or just hit 'q' on the first latex
% run, let it finish, and you should be clear).
\usepackage[breaklinks=true,bookmarks=false]{hyperref}
\hypersetup{
	colorlinks   = true, %Colours links
	urlcolor     = blue, %Colour for external hyperlinks
	linkcolor    = blue, %Colour of internal links
	citecolor   = green %Colour of citations
}

\cvprfinalcopy 

\def\httilde{\mbox{\tt\raisebox{-.5ex}{\symbol{126}}}}

\setcounter{page}{1}
\begin{document}

%%%%%%%%% TITLE
\title{\LaTeX\ Guidelines for Uplinx Seminar Final Report}

\author{First Author\\
{\tt\small firstauthor@i1.org}
\and
Second Author\\
{\tt\small secondauthor@i2.org}
\and
Third Author\\
{\tt\small thirdauthor@i3.org}
}

\maketitle
%\thispagestyle{empty}

%%%%%%%%% ABSTRACT
\begin{abstract}
   The ABSTRACT is to be in fully-justified italicized text, at the top
   of the left-hand column, below the author information. Use the word 
   ``Abstract'' as the title, in 12-point Times, boldface type, centered 
   relative to the column, initially capitalized. The abstract is to be 
   in 10-point, single-spaced type. It should be a concise summary of 
   your final report. Limit yourself to about 150 words. 
   Leave two blank lines after the Abstract, then begin the main text.   
   
\end{abstract}

%%%%%%%%% BODY TEXT
\section{Report Content}
These guidelines will help you structure the final report of your practical 
assignment. The template is based on the Author Guidelines for CVPR 
proceedings. Follow these guidelines when writing your report. No other 
format will be accepted.



%-------------------------------------------------------------------------
\subsection{Structure}
Your report should at least include the following:
\begin{itemize}
	\itemsep-0.4em 
	\item \textbf{Title}: A few words that describe your project.
	\item \textbf{Abstract}: A brief summary of the paper that tells your readers what they can expect. Include the motivation, problem statement, how you solve it, results, and conclusion.
	\item \textbf{Introduction}: Give a motivation and state what the problem is. You can talk about convolutional neural networks, the domain of autonomous driving, etc.
	\item \textbf{Related Work}: What other people have done in the past in relation to what you are doing.
	\item \textbf{Method}: How you propose to solve the problem. This is the main section of your paper. Describe what you have done and how you solved the main task: the architecture, preprocessing, etc. Describe also what other subtasks you undertook: effect of transfer learning, amount of data necessary, architecture changes, hyperparameters, etc.
	\item \textbf{Results}: Evaluation of your method. Describe the dataset, metrics, experiment design, etc.
	\item \textbf{Discussion}: Analysis of the results, limitations, etc. How do your results match up to the stated problem.
	\item \textbf{Conclusion}: Wrap up. Restate what you did, what you achieved and what needs to be done in the future. 
	\item \textbf{References}: A list of the works you cite.
\end{itemize}
If it makes more sense to restructure the paper a little, feel free to do so, but make sure you still cover all this information somewhere.

\begin{figure}[t]
	\begin{center}
		\fbox{\rule{0pt}{2in} \rule{0.9\linewidth}{0pt}}
		%\includegraphics[width=0.8\linewidth]{egfigure.eps}
	\end{center}
	\caption{Example of a figure. The caption is set in Roman so that mathematics
		(always set in Roman: $B \sin A = A \sin B$) may be included without an
		ugly clash.}
	\label{fig:long}
	\label{fig:onecol}
\end{figure}

\begin{figure*}
	\begin{center}
		\fbox{\rule{0pt}{2in} \rule{.9\linewidth}{0pt}}
	\end{center}
	\caption{This figure occupies both columns. Short caption is centered.}
	\label{fig:short}
\end{figure*}


\subsection{Language}

All manuscripts must be in English.

\subsection{Paper length}
Your paper should be at least 4 pages long and no longer than 8.

%-------------------------------------------------------------------------

\subsection{Mathematics}
You might want to use equations to explain certain concepts. 
\begin{equation} \label{eq:1}
e^{\pi i} + 1 = 0
\end{equation}
Please number all of your sections and displayed equations.  It is
important for you and your readers to be able to refer to any particular 
equation, like equation \ref{eq:1} in this case. 


\subsection{Figures}
Just like math, images also help the reader understand what you are talking
about. You can reference them just like equations. As in Figure \ref{fig:onecol}, 
make sure you use captions. Captions should tell the reader what he's looking at
without needing to lookup where in the text you talk about the image. Don't 
overdo it though, explain the image concisely. 

To include an image it's almost always best to use \verb+\includegraphics+, and 
to specify the  figure width as a multiple of the line width as in the example
below
{\small\begin{verbatim}
	\usepackage[dvips]{graphicx} ...
	\includegraphics[width=0.8\linewidth]
	{myfile.eps}
	\end{verbatim}
}







Another possibility is to use tables, which may come in handy when presenting 
and discussing the results.


\begin{table}[H]
	\begin{center}
		\begin{tabular}{|l|c|}
			\hline
			Method & Frobnability \\
			\hline\hline
			Theirs & Frumpy \\
			Yours & Frobbly \\
			Ours & Makes one's heart Frob\\
			\hline
		\end{tabular}
	\end{center}
	\caption{Results.   Ours is better.}
	\label{tab:1}
\end{table}

\section{Formatting your paper}


%------------------------------------------------------------------------


This section introduces some important aspects on how to format your report 
correctly.

%-------------------------------------------------------------------------
\subsection{Type-style and fonts}

Wherever Times is specified, Times Roman may also be used. If neither is
available on your word processor, please use the font closest in
appearance to Times to which you have access.

MAIN TITLE. Center the title 1-3/8 inches (3.49 cm) from the top edge of
the first page. The title should be in Times 14-point, boldface type.
Capitalize the first letter of nouns, pronouns, verbs, adjectives, and
adverbs; do not capitalize articles, coordinate conjunctions, or
prepositions (unless the title begins with such a word). Leave two blank
lines after the title.

AUTHOR NAME(s) are to be centered beneath the title
and printed in Times 12-point, non-boldface type. This information is to
be followed by two blank lines.

\textbf{All text must be in a two-column format}. Long figures may occupy 
both columns as in Figure \ref{fig:short}, but short captions should be 
centered. 
Each section should have some text, don't just introduce a subsection 
directly below a section without first explaining what the section is about.

MAIN TEXT. Type main text in 10-point Times, single-spaced. Do NOT use
double-spacing. All paragraphs should be indented 1 pica (approx. 1/6
inch or 0.422 cm). Make sure your text is fully justified---that is,
flush left and flush right. Please do not place any additional blank
lines between paragraphs.

Figure and table captions should be 9-point Roman type as in
Figures~\ref{fig:onecol}, ~\ref{fig:short} and Table ~\ref{tab:1}.


%-------------------------------------------------------------------------
\subsection{Footnotes}

Please use footnotes\footnote {This is what a footnote looks like.  It
often distracts the reader from the main flow of the argument.} sparingly.
Indeed, try to avoid footnotes altogether and include necessary peripheral
observations in the text (within parentheses, if you prefer, as in this sentence).
If you wish to use a footnote, place it at the bottom of the column on the 
page on which it is referenced. Use Times 8-point type, single-spaced.


%-------------------------------------------------------------------------
\subsection{References}
List and number all bibliographical references in 9-point Times,
single-spaced, at the end of your paper. When referenced in the text,
enclose the citation number in square brackets, for
example~\cite{Authors14}.  Where appropriate, include the name(s) of
editors of referenced books.

When citing a multi-author paper, you may save space by using ``et alia'',
shortened to ``\etal'' (not ``{\em et.\ al.}'' as ``{\em et}'' is a complete word.)
However, use it only when there are three or more authors.  Thus, the
following is correct: ``
Frobnication has been trendy lately.
It was introduced by Alpher~\cite{Alpher02}, and subsequently developed by
Alpher and Fotheringham-Smythe~\cite{Alpher03}, and Alpher \etal~\cite{Alpher04}.''

This is incorrect: ``... subsequently developed by Alpher \etal~\cite{Alpher03} ...''
because reference~\cite{Alpher03} has just two authors.  If you use the
\verb'\etal' macro provided, then you need not worry about double periods
when used at the end of a sentence as in Alpher \etal.





{\small
\bibliographystyle{ieee}
\bibliography{egbib}
}

\end{document}
